\documentclass[11pt, notitlepage]{article}
\usepackage[utf8]{inputenc}

% \usepackage[nottoc]{tocbibind}
\usepackage{datetime}
\usepackage{graphicx}
\usepackage[nointegrals]{wasysym}
\usepackage{mathtools}
\usepackage{tikz}
\usepackage{tikz-cd}
\usepackage{marginnote}
\usepackage{microtype} % Slightly tweak font spacing for aesthetics
\usepackage{amsmath, amsfonts, amssymb, amsthm}
\usepackage{bbm}
\usepackage{listings}
% \usepackage{thm-autoref}
\usepackage{hyperref}
\hypersetup{
  colorlinks   = true, %Colours links instead of ugly boxes
  urlcolor     = blue, %Colour for external hyperlinks
  linkcolor    = blue, %Colour of internal links
  citecolor   = red %Colour of citations
}

\newtheorem{thm}{Theorem}[section]

\newtheorem{lem}[thm]{Lemma}
% \newcommand{\lemautorefname}{Lemma}

\newtheorem{claim}[thm]{Claim}
\newtheorem{prop}[thm]{Proposition}
\newtheorem{cor}[thm]{Corollary}
\newtheorem{conj}[thm]{Conjecture}

\theoremstyle{definition}
\newtheorem{defn}[thm]{Definition}

\theoremstyle{definition}
\newtheorem{example}[thm]{Example}

\theoremstyle{definition}
\newtheorem*{question}{Question}
\newtheorem*{exercise}{Exercise}
\newtheorem{probstate}{Problem}

\theoremstyle{remark}
\newtheorem*{rem}{Remark}

\usepackage[top=2cm, bottom=2cm, inner=2cm, outer=4cm, marginparwidth=3.4cm]{geometry}

\usepackage{gentium}
%\usepackage[euler-digits,small]{eulervm}

\usepackage{enumerate}
\usepackage{xcolor}

%-------------------------------
%       Custom environments and counters
%-------------------------------

%\newcounter{problemnum}
%\setcounter{problemnum}{1}

\newenvironment{problem}[2]{
    \begin{probstate}[Date: #1, tags: {\color{white} \texttt{#2}}]
}
{
  \end{probstate}
}

\newenvironment{solution}{
    \begin{proof}[Solution]
}
{
    \end{proof}
}

\newenvironment{sketch}{
    \begin{proof}[Solution sketch]
}
{
    \end{proof}
}

\title{Real Analysis qualifying review big list}
\author{Students at the University of Michigan}
% \date{\formatdate{3}{10}{2019}}

\makeatletter
\makeatother

\newcommand{\R}{\mathbb{R}}
\newcommand{\Z}{\mathbb{Z}}
\newcommand{\N}{\mathbb{N}}
\newcommand{\norm}[1]{{\left\|#1\right\|}}

\begin{document}
\maketitle

\tableofcontents

\section*{How to add to this document}
This document contains a subset of the old real analysis qual problems, and solutions to some of them.
To add a problem to this list, use the \texttt{problem} environment: this environment takes two arguments, the \texttt{date} the problem appeared on a qual, and \texttt{tag}s describing the problem.
\begin{lstlisting}[language=TeX]
\begin{problem}{<Month> <Year>}{tag1, tag2, ...}
  <Problem statement>
\end{problem}
\end{lstlisting}

To add a solution, or a solution sketch, use the \texttt{solution} and \texttt{sketch} environments.
These environments do not take any arguments.

\subsection*{Notation}
Unless otherwise specified, $m$ refers to the Lebesgue measure on $\mathbb{R}^n$ and subsets of $\mathbb{R}^n$, and $m^{\ast}$ refers to the Lebesgue outer measure.

\section{Categorized (without solutions)}

\subsection{Basic measure theory and integrability}


\begin{problem}{2013 draft}{basic measure theory}
Let $(X, \mathcal{A},\mu)$ be a finite measure space. For a set $A \subset X$ define $\mu_*(A)=\mu(X)-\mu^*(X \setminus A)$, where $\mu^*$ is the outer measure. Prove that $\mu_*(A) \le \mu^*(A)$ for any $A \subset X$.
\end{problem}


\begin{problem}{2013 draft}{convergence in measure}
Let $f_1, f_2, \ldots, f, g$ be measurable functions on a measure space  $(X, \mathcal{A},\mu)$. Assume that $f_n \to f$ in measure and $f_n \le g$ a.e. Prove that $f \le g$ a.e.
\end{problem}

\begin{problem}{2013 draft}{basic measure theory}
Let $\{x_n\}_{n=1}^{\infty} \subset [0,1]$ be any sequence. For $n \in \N$ define the set $A_n \subset \R$ by
  \[
   A_k=\bigcup_{n=k}^{\infty} \left(x_n-\frac{k}{n^3}, x_n+ \frac{k}{n^3} \right).
  \]
  Prove that $m(\bigcap_{k=1}^{\infty} A_k)=0$, where $m$ denotes the Lebesgue measure.
\end{problem}


\begin{problem}{September 2014}{integrability?}
Let $\{f_k(x)\}$ be a sequence of nonnegative measurable functions on $E$ and $m(E)<\infty$. Show that $\{f_k(x)\}$ converges in measure to $0$ if and only if
$$\lim_{k\to\infty}\int_E\frac{f_k(x)}{1+f_k(x)}\,dx=0.$$
\end{problem}


\begin{problem}{September 2014}{fat Cantor set}
Construct a measurable subset $A$ of $(0,1)$ such that $m(A)<1$ and
 $m \big(A \cap (a,b) \big) >0$ for any $(a,b) \subset (0,1)$.
\end{problem}

\begin{problem}{January 2015}{Carath\'eodory's criterion}
Let $A, B\subset \mathbb R^d$. Assume $A\cup B$ is measurable, and $m(A\cup B)<\infty$. If
$$m(A\cup B)=m^*(A)+m^*(B)$$
Show that $A$ and $B$ are measurable.

(Hint: prove first that for any set $A$, there a measurable set $U$, with $A\subset U$, such that $m^*(A)=m(U)$.)
\end{problem}

\begin{problem}{January 2015}{simple function approximation}
Let $f$ be a nonnegative measurable function on $(0,1)$. Assume that there is a constant $c$, such that
$$\int_0^1 (f(x))^n\,dx=c,\qquad n=1,2,\dots$$
Show that there is a measurable set $E\subset (0,1)$, such that
$$f(x)=\chi_E(x),\qquad \text{ for a.e. }  x\in(0,1).$$
\end{problem}

\begin{problem}{May 2020}{$L^p$ spaces}
Suppose $f$ is a $C^1$ function on $\mathbb{R}$ satisfying $f(0)=0, \ |f(x)|\le |x|^{-1/2}, \ x\ne 0$. Let $g$ be in  $L^1(\mathbb{R})$.
\begin{enumerate}[(a)]
\item Show there is a constant $C$ such that $m\{|g|>\alpha\}\le C/\alpha$ for all $\alpha>0$.
\item Show that the function $h(x)=f(g(x))$ is in $L^1(\mathbb{R})$.
\end{enumerate}
\end{problem}

\begin{problem}{May 2020}{triangle inequality?}
  Let $r_n, \ n=1,2,\ldots,$ be an enumeration of the rationals in the interval $[0,1]$ and consider the function  $f:[0,1]\to\mathbb{R}\cup{\infty}$ defined by
$$
f(x) \ = \ \sum_{n=1}^\infty \frac{1}{n^2}\frac{1}{|x-r_n|^{1/3}} \ , \quad 0\le x\le 1 \ .
$$
Show that $f\in L^2(0,1)$.
\end{problem}

\begin{problem}{May 2020}{outer regularity}
  Let $E \subset (0,1)$ be a measurable set such that for any interval $(a,b) \subset (0,1)$, there exists an interval $(c,d) \subset (a,b) \setminus E$ with
 \[
  d-c \ge \frac{a}{10} (b-a).
 \]
 Prove that $m(E)=0$.

\end{problem}

\begin{problem}{September 2019}{countable subadditivity?}
  Let $E$ be the set of all $x\in(0,1)$ such that there exists  a sequence of irreducible fractions
$\{p_n/q_n\}_{n\in\mathbb{N}}$ with $p_n,q_n\in\mathbb{N}, \ q_1<q_2<\cdots$ such that
$$
\left|x-\frac{p_n}{q_n}\right| \ \le \ \frac{1}{q_n^3} \ , \quad n=1,2,...
$$
Prove that the Lebesgue measure of $E$ is zero.
\end{problem}

\begin{problem}{January 2012}{$L^p$ spaces}
  Construct a function $f \in L_1(\R)$ such that $f \notin L_2((a,b))$ for any interval $(a,b) \subseteq \R$.
\end{problem}

\begin{problem}{September 2011}{measurable sets?}
  Let $f \in L_1([0,1])$ be a function such that $\int_E f (x) \, dx=0$ for any measurable set $E \subset [0,1]$ of Lebesgue measure $1/2$.
  Prove that $f=0$ a.e.
\end{problem}

\begin{problem}{January 2011}{measurable functions}
  \item Let A be a sequence of measurable subsets of $[0,1]$ such that $\inf m(A_n)>0$, where $m$ stands for the Lebesgue measure.
   \begin{enumerate}[(i)]
    \item Prove that there exists $x \in [0,1]$ which belongs to infinitely many of the sets $A_n$.
    \item Does there necessarily exist a point which {\color{red} (does not?)} belong to any of the sets $A_n$, except finitely many?
   \end{enumerate}
\end{problem}

\begin{problem}{January 2011}{simple function approximation}
  Let $f:X \to [0, +\infty)$ be an integrable function on
    a measure space $(X, \mathcal{A},\mu)$.
    Define the measure $\nu$ by $\nu(A)=\int_A f \, d \mu$.
    \begin{enumerate}[(i)]
        \item Prove that the measure $\nu$ is $\sigma$-additive.

        \item Prove that if
        $g \in L_1(\nu)$, then
        $\int_X g \, d \nu =\int_X fg \, d \mu$.

        (Hint: first, assume that $g$ is a simple positive
        function. Then extend the result to non-negative
        integrable functions using limit theorems).
    \end{enumerate}
\end{problem}

\begin{problem}{January 2011}{simple functions}
  Let $f:X \to [0, \infty]$ be a measurable
 function. Assume that $\mu (X) < \infty$.
 Prove that $\int f \, d \mu< +\infty$ if and only if
 \[
 \sum_{n=1}^{\infty} 2^n \mu (x \in X \mid f(x) \ge 2^n) < +\infty.
 \]
\end{problem}

\begin{problem}{January 2012}{measurable functions}
  Let $f : [a,b] \to \R$ be a differentiable function.  Prove that the function $f'$ is measurable.
\end{problem}

\begin{problem}{January 2012}{$L^p$ spaces}
  Let $1 \le p < \infty$, and let $f \in L_p(\mu)$. Prove that
\[
 \lim_{t \to 0} t^p \mu \{x \mid |f(x)|>t \} =0.
\]
\end{problem}

\subsection{Integral convergence}

\begin{problem}{2013 draft}{integral convergence}
Prove that
  \[
   \lim_{n \to \infty} \int_\R \frac{\cos^{n} (\pi  x)}{(x-n)^2+1} \, dx
  \]
  exists and find it.
\end{problem}

\begin{problem}{May 2011}{integral convergence}
  Let $f_n,g_n,f,g \in L_1(\mu)$ be functions such that
    $f_n \to f$ a.e., $g_n \to g$ a.e. and $|f_n| \le g_n$. Prove
    that if $\int g_n \, d \mu \to \int g \, d \mu$, then
    $\int f_n \, d \mu \to \int f \, d \mu$.
    (Hint: use Fatou's Lemma.)
\end{problem}

\begin{problem}{September 2019}{integral convergence}
  Let  $f$ be a measurable function on $(0,\infty)$, and for $n=1,2,\ldots$ let
$f_n$ be defined by
$$
f_n(x)=f(x)e^{-x}\left[1+x+\frac{x^2}{2!}+\cdots+\frac{x^n}{n!}\right] \ .
$$
Suppose $f \in L^2[(0,\infty)]$. Prove that $\lim_{n\to \infty}\|f_n-f\|_{L^2[(0,\infty)]}=0$
\end{problem}

\begin{problem}{January 2012}{integral convergence}
  Let $(X, \mathcal{A}, \mu)$ be a finite measure space ($\mu(X)< \infty$). Assume that a sequence $\{f_n\}_{n=1}^{\infty} \subseteq L_1(\mu)$ satisfies the condition
\[
  \frac{1}{\sqrt{\mu(E)}} \int_E |f_n| \, d\mu \le 1
\]
for all $n \in \N$ and all sets $E$ of positive measure. Prove that if $f_n \to f$ a.e., then $f \in L_1(\mu)$ and
\[
  \int_X f_n \, d \mu \to \int_X f \, d \mu.
 \]
\end{problem}

\begin{problem}{January 2011}{integral convergence}
  Let $\{f_n\} \subset L_1(\mu)$ be a decreasing sequence of functions such that $f_n \to f$ a.e.
  Prove that
    \[
      \lim_{n \to \infty} \int f_n \, d \mu = \int f \, d \mu.
    \]
\end{problem}

\begin{problem}{September 2011}{$L^p$ spaces}
  For a real valued function $f(x,y)$ on $\mathbb R^2$ which is in $L^2$, show that $f(x+\epsilon,y+\epsilon) \rightarrow f(x,y)$ in $L^2$ when $\epsilon \rightarrow 0$.
\end{problem}

\begin{problem}{January 2011}{}
  Let $f_n,f \in L_1(\mu)$ be such that $f_n \to f$ a.e.
    Prove that if \newline
    $\|f_n\|_1 \to \|f\|_1$,
    then $f_n \to f$ in $L_1(\mu)$.

    (Hint: use the previous problem)

\end{problem}

\subsection{Integral inequalities}

\begin{problem}{January 2014}{Hardy-Littlewood maximal inequality}
Let $E \subset [0,1]$ be a measurable set, $m(E) \ge \dfrac{99}{100}$. Prove that there exists $x \in [0,1]$ such that for any $r \in (0,1)$,
    \[
      m \big( E \cap (x-r,x+r) \big) \ge \frac{r}{4}.
    \]

\emph{Hint:} One approach to this problem involves the Hardy-Littlewood maximal inequality.
\end{problem}

\begin{problem}{January 2014}{H\"older's inequality, change of variables?}
Find all $q \ge 1$, such that $f(x^2) \in L_q((0,1),m)$ for any $f(x)  \in L_4((0,1),m)$, where $m$ denotes the Lebesgue measure.
\end{problem}

\begin{problem}{September 2014}{H\"older's inequality}
Let $K=\{f :(0,+\infty) \to \R \mid \int_0^{\infty} f^4(x) \, dx \le 1  \}$. Evaluate
\[
  \sup_{f \in K} \int_0^\infty f^3(x) e^{-x}  \, dx.
\]
\end{problem}

\begin{problem}{September 2019}{Fubini, Hölder}
    Let $f:\mathbb{R}\times (0,1)\to\mathbb{R}$ be a measurable function such that for any $y\in(0,1)$,
$$\int_{\mathbb{R}} f^2(x,y) \ dx \ \le \ 1\  .
$$
Prove there exists a sequence $\{x_n\}_{n\in\mathbb{N}}$, with $\lim_{n\to\infty}x_n=+\infty$, such that
$$\lim_{n\to\infty} \int_0^1 |f(x_n,y)| \ dy \ = \ 0 \  .
$$
\end{problem}

\begin{problem}{January 2011}{integral inequalities?}
  Let $f \in L_{\infty}([0,1])$. Prove that
 \[
   \lim_{n \to \infty} \frac{\int_{[0,1]} |f(x)|^{n+1} \, dx}{\int_{[0,1]} |f(x)|^{n} \, dx} = \|f\|_{\infty}.
 \]
\end{problem}

\begin{problem}{January 2011}{integral inequalities}
  Let $f_n: \mathbb{R} \to [0, 1]$ be functions such that
 $\sup_{x \in \mathbb{R}} f_n(x)=1/n$ and $\int_{\mathbb{R}} f(x) \, dx=1$. Set
 \[
  F(x)= \sup_{n \in \mathbb{N}} f_n(x).
 \]
 Find all possible values of $\int_{\mathbb{R}} F(x) \, dx$.

\end{problem}

\begin{problem}{September 2011}{hölder's inequality}
  Let $(X,\Omega, \mu)$ be a finite measure space.
 \begin{enumerate}[(i)]
   \item Prove that for any $p<q, \ L_q(\mu) \subset L_p(\mu)$.
   \item Assume that for any
 $t>0$ there exists $E \in \Omega$ satisfying
 \[
  0< \mu(E) <t.
 \]
 Prove that for any $1<p< \infty$ there exists a function $f \in
 L_p(\mu)$ such that $f \notin L_q(\mu)$ for any $q>p$.
\end{enumerate}
\end{problem}


\subsection{Miscellaneous}

\begin{problem}{September 2011}{bounded variation}
  Suppose $f:[0,1] \to \R$ satisfies $f(x)-f(y)<x-y$ for all $x,y\in [0,1], x>y$.
  Show that $f'$ exists almost everywhere on $[0,1]$ or give a counterexample.
\end{problem}


\begin{problem}{January 2011}{egoroff's theorem}
  Let $E$ be the exceptional set in Egoroff's theorem.
    Is it possible to prove Egoroff's theorem with
    $\l(E)=0$ instead of $\l(E) < e$?
\end{problem}


\begin{problem}{January 2011}{change of variables}
  Let $f,g: \mathbb R \rightarrow \mathbb R$ be $L_1$-functions.
  \begin{enumerate}[(a)]
\item Prove that
\[
\int_{\mathbb R} |f(x-y)g(y)| dm(y) < +\infty.
\]
  \item Let
\[
h(x)= \int_{\mathbb R} f(x-y)g(y) dm(y).
\]
Prove that $h \in L_1(\mathbb R)$ and $\|h\|_1 \le \|f\|_1 \cdot \|g\|_1$.
\end{enumerate}
\end{problem}

\begin{problem}{January 2012}{Lebesgue differentiation theorem, Hardy-Littlewood maximal estimate?}
  Let  $f \in L_1(\R)$.
  For $n \in \N$ define the function $g_n: \R \to \R$ as follows.  For $k \in \Z$ and for $x \in [k/n,(k+1)/n)$ set
\[
 g_n(x)=n \int_{k/n}^{(k+1)/n} f (x) \, dx.
\]
Prove that $g_n$ converges to $f$ a.e. and in $L_1(\R)$.
\end{problem}


\begin{problem}{September 2019}{absolute continuity}
   A function $f:(0,1)\to\mathbb{R}$ is locally Lipschitz if for   any $x\in(0,1)$ there is an open interval $I_x$ with $x\in I_x\subset (0,1)$ and a constant $C_x$ such that $|f(y)-f(y')|\le C_x|y-y'|$ for $y,y'\in I_x$.

\begin{enumerate}[(a)]
\item Prove that a locally Lipschitz function $f(\cdot)$ is absolutely continuous on any compact subinterval
$[a,b]\subset (0,1)$.

\item  Give an example of a locally Lipschitz function $f:(0,1)\to\mathbb{R}$ which extends to a continuous function on the closed interval $[0,1]$, but is not absolutely continuous on $[0,1]$.
\end{enumerate}
\end{problem}

\begin{problem}{May 2020}{bounded variation}
  Let $f:\mathbb{R}\to\mathbb{R}$ be a Lebesgue measurable function such that
$$
f(y) \ \le \ f(x)+(x^2+y^2)(x-y) \quad {\rm for \ } -\infty<y<x<\infty \ .
$$
Show that the derivative function $x\to f'(x)$ exists a.e. on $\mathbb{R}$.
\end{problem}

\begin{problem}{May 2011}{Egorov's theorem}
  Let $\{f_n: [0,1] \to \R\}_{n=1}^\infty$ be a sequence of continuous functions
     such that $f_n(x) \to f(x)$ for any $x \in [0,1]$. Does there
     exist a set $E \subset [0,1]$ of Lebesgue measure 0 such that
     $f_n \to f$ uniformly on $[0,1] \setminus E$?
\end{problem}

\begin{problem}{January 2011}{}
  Let $f \in L_1([0,1])$ be a function such that $f(x)>0$ a.e.
 \begin{enumerate}[(i)]
 \item Prove that for any $0<a<1$
 \[
   \inf_{m(A)=a} \int_A f \, dm >0.
 \]
 \item Does the previous statement hold for a function $f \in L_1(\mathbb R)$ such that $f(x)>0$ a.e.
 \end{enumerate}
\end{problem}

\begin{problem}{January 2015}{Egorov's theorem}
Let $E_k\subset [a,b]$, $k\in\mathbb N$ be measurable sets, and there exists $\delta>0$ such that $m(E_k)\ge \delta$ for all $k$.
 Assume that $a_k\in\mathbb R$ satisfies
$$\sum_{k=1}^\infty |a_k|\chi_{E_k}(x)<\infty\qquad \text{ for a.e. }  x\in [a,b].$$
Show that
$$\sum_{k=1}^\infty |a_k|<\infty.$$
(For extra challenge, find a proof that does not use Egorov's theorem).
\end{problem}

\begin{problem}{September 2014}{Lebesgue differentiation}
Let  $f: \R \to \R$ be a function such that $\int_{\R} |f(x)| \, dx < \infty$. Show that the sequence
\[
 h_n(x)=\frac{1}{n} \sum_{k=1}^n f \left(x+ \frac{k}{n} \right)
\]
converges in $L_1(\R)$.
\end{problem}

\begin{problem}{January 2014}{Dominated convergence}
Let $f\in L_1\cap L_4$ (on some measure space).  Prove that the function defined on $[1,4]$, given by the following formula
\begin{align*}
p & \mapsto \|f\|_p
\end{align*}
 is continuous.
\end{problem}

\begin{problem}{January 2014}{Fubini}
Let
\begin{align*}
E &\subset \{(x,y) \mid 0 \le x \le 1, \ 0 \le y \le x\}\\
E_x &= \{y\mid (x,y)\in E\}\\
E_y &= \{x\mid (x,y)\in E\}
\end{align*}
and assume that $m(E_x) \ge x^3$ for any $x \in [0,1]$.
\begin{enumerate}[(i)]
 \item Prove that there exists $y \in [0,1]$ such that $m(E_y) \ge \dfrac{1}{4}$.
 %\medskip
 \item (Hard) Prove that there exists $y \in [0,1]$ such that $m(E_y) \ge c$, where $c>1/4$ is a constant independent of $E$. Find the optimal such $c$.
 %optimal value 1-\dfrac{1}{\sqrt{2}}
\end{enumerate}
\end{problem}

\section{Categorized (with solutions)}

\begin{problem}{September 2014}{integrability, Fubini}
Let $1\le p<\infty$, $f\in L^p(\mathbb R^n)$. Let
$$f_*(\lambda)=m(\{x: |f(x)|>\lambda\}), \quad \lambda>0$$
Show that
\begin{enumerate}[(i)]
\item$p\int_0^\infty \lambda^{p-1} f_*(\lambda)\,d\lambda= \int |f(x)|^p\,dx$
\item  $\lim_{\lambda\to\infty}\lambda^p f_*(\lambda)=0$
\item $\lim_{\lambda\to 0}\lambda^p f_*(\lambda)=0$
\end{enumerate}
\end{problem}
\begin{sketch}
For simplicity, prove the result for $L^1$ functions, and use a change of variables argument to prove it for $p > 1$.
For part (i), express $f_{\ast}(\lambda)$ as the integral over $\mathbb{R}$ of the indicator of $\left\{x \mid |f(x)| > \lambda \right\}$, and then use Fubini to swap integrals.
For parts (ii) and (iii), note that $f_{\ast}$ is decreasing, so it will suffice to prove it for a sequence of $\lambda$ going to $\infty$ and $0$.
We pick a sequence such that the sums of corresponding quantities are bounded above by an absolute constant times the integral of $f$: we can do this by interpreting the quantities in the sequence as areas of rectangles under the graph of the function $f$.
\end{sketch}

\begin{problem}{January 2014}{\color{red} Unclear what the tag should be}
Prove or disprove:  If $E$ is an open subset of $\R$ with $m(E)=1$ then there is a finite union of intervals $F$ containing $E$ with $m(F)<1.1$.
\end{problem}
\begin{sketch}
False: consider the set $E$ obtained by taking the unions of balls of radius $\dfrac{1}{n^2}$ around $n \in \mathbb{N}$.
This open set has finite measure, but any finite union of intervals containing it will have infinite measure.
\end{sketch}

\begin{problem}{2013 draft}{basic measure theory}
Let $A \subset [0,1] \times [0,1]$ be the set of points $(x,y)$ with decimal representations $x=0.x_1 x_2 \ldots, \ \  y=0.y_1 y_2 \ldots$ such that  $x_n y_n=5$ for all $n \in \N$. Prove that the set $A$ is measurable and find its Lebesgue measure.
\end{problem}
\begin{sketch}
Observe that for any $(x,y) \in A$, $x+y = 0.\overline{6}$. This means $A$ is contained in the graph of the continuous function $y = 0.\overline{6} - x$, which is a measure $0$ set.
Subsets of measure $0$ sets are measurable, and have measure $0$.
\end{sketch}

\begin{problem}{September 2011}{integral convergence}
Let $f \in L_1([0,1], dx)$. Find
 \[
   \lim_{n \to \infty} \frac{1}{n}  \int_0^1 \log \left( 1+ e^{nf(x)} \right) \, dx.
 \]
\end{problem}
\begin{sketch}
Step 1: Consider pointwise convergence in two different sets: $f(x) \leq 0$ and $f(x) > 0$.
Step 2: Use convergence theorems in these two domains to get convergence of integral.
\end{sketch}

\begin{problem}{May 2020}{change of variables, integral convergence}
  Let $f_n, \ n=1,2,...,$ be the sequence of functions on $(0,\infty)$ defined by
$$
f_n(x) \ = \ \frac{1}{n}\left(1-\frac{x}{n}\right)^n e^{x} \ ,  \ \ 0<x<n, \quad f_n(x)=0, \ \ x\ge n.
$$
Prove that the sequence $a_n, \ n=1,2,\dots,$ given by
$$
a_n \ = \ \int_0^\infty f_n(x) \ dx \quad {\rm converges \ and \ identify \ } a_\infty=\lim_{n\to\infty} a_n.
$$
\end{problem}

\begin{sketch}
Step 1: Change variables so that all integrals are over $[0,1]$.
Step 2: Observe that the new integrand is converging pointwise to $0$ a.e.
Step 3: Use monotone/dominated convergence theorem.
\end{sketch}


\begin{problem}{January 2015}{h\"older's inequality}
Let $f$ be locally integrable on $\mathbb R^n$, $1<p<\infty$. Show that the following are equivalent:

\begin{enumerate}[(i)]
    \item $f\in L^p(\mathbb R^n)$.
    \item there exist $M>0$, such that for any finite collection of mutually disjoint measurable sets $E_1, E_2,\dots, E_k$, with $0<m(E_i) <\infty$ for $1\le i\le k$,
$$\sum_{i=1}^k \left(\frac1{m(E_i)}\right)^{p-1}\left|\int_{E_i} f(x)\,dx\right|^p\le M.$$
\end{enumerate}
\end{problem}

\begin{sketch}
  For $(i) \implies (ii)$, use H\"older's inequality with $f$ and the indicator functions of $E_i$. For $(ii) \implies (i)$, observe that the inequality in $(ii)$ implies that $\int_E |fg| \leq M \norm{g}_q$ for all $g \in L^q(E)$. This follows from approximating $g$ by simple functions.
  The inequality shows that integration against $f$ is a bounded linear functional on $L^q$, and therefore, $f$ must be in $L^p$ by the Riesz representation theorem.
\end{sketch}


\begin{problem}{September 2019}{Hölder}
  Let $(X,\Omega,\mu)$ be a measure space with $\mu(X)=1$, and let $f\in L^2(\mu)$ be a non-negative function satisfying $\int_X f \ d\mu\ge 1$. Prove that
$$
\mu\left(\{x\in X  \mid \ f(x)>1\}\right) \ \ge \  \frac{\left(\int_{X} f \ d\mu-1\right)^2}{\int_X f^2 \ d\mu} \ .
$$
\end{problem}

\begin{sketch}
Use Cauchy-Schwartz with $f$ and the indicator of the set where $f(x) > 1$.
\end{sketch}

\section{Uncategorized}

\begin{problem}{2013 draft}{}
Let $\mu_1 \le \mu_2 \le \ldots$ be a sequence of
    positive absolutely continuous measures on a measure space $(X,
    \mathcal{A},\rho)$. Assume that there exists a finite positive measure
    $\nu$ such that $\mu_n \le \nu$ for all $n\in \N$. For $A \in
    \mathcal{A}$ set $\mu(A) = \lim_{n \to \infty} \mu_n(A)$. Prove
    that $\mu$ is an absolutely continuous measure.

    (Hint: use Lebesgue--Radon--Nikodym Theorem.)
\end{problem}

\begin{problem}{2013 draft}{}
Let $f_1, f_2, \ldots,f: [0,1] \to \R$ be non-decreasing functions such that $\sum_{n=1}^{\infty} f_n =f$. Prove that $\sum_{n=1}^{\infty} f_n' =f'$ a.e.
\end{problem}

\begin{problem}{2013 draft}{}
(Hard) Prove that the sequence
    \[
     f_n(x)= n^{1/2} \exp \left(- \frac{n^2x^2}{x+1} \right)
    \]
    converges in $L_p([0,+\infty))$ for $1 \le p<2$ and diverges for $p \ge 2$.
\end{problem}

\begin{problem}{2013 draft}{}
Let  $(X, \mathcal{A},\mu)$ be a $\sigma$-finite measure space with $\mu(X)=\infty$. Construct a function $F: X \to \R$ such that $F \in L_p(\mu)$ for all $p>1$, but $F \notin L_1(\mu)$.
\end{problem}

\begin{problem}{2013 draft}{}
(Hard?) Let $K=\{f :(0,+\infty) \to \R \mid \int_0^{\infty} f^4(x) \, dx \le 1  \}$. Evaluate
\[
  \sup_{f \in K} \int_0^\infty \frac{f^3(x)}{1+x}  \, dx.
\]
\end{problem}


\begin{problem}{2013 draft}{}
\begin{enumerate}[(i)]
   \item (Easy) Let $E \subset [0,1]$ be a measurable set, $m(E) \ge \frac{99}{100}$. Prove that there exists $x \in [0,1]$ such that for any $r \in (0,1)$,
    \[
      m \big( E \cap (x-r,x+r) \big) \ge \frac{r}{4}.
    \]
    \item (Hard) Let $E \subset [0,1]$ be a measurable set, $m(E) \ge \frac{1}{2}$. Prove that there exists $x \in [0,1]$ such that for any $r \in (0,1)$,
    \[
      m \big(E \cap (x-r,x+r) \big) \ge \frac{r}{20}.
    \]
\end{enumerate}
\end{problem}

\begin{problem}{2013 draft}{}
Find all $q \ge 1$, such that $f(x^2) \in L_q((0,1))$ for any $f  \in L_4((0,1))$.
\end{problem}

\begin{problem}{2013 draft}{}
Let $E_n, \ n \in \N$ be measurable sets. Prove that the set of $x \in \R$ for which there exists at most 3 values of $n$ such that $x \in E_k$, but $x \notin E_{k^n}$ for all $n \in \N \setminus \{1\}$ is measurable.
\end{problem}

\begin{problem}{2013 draft}{}
(Hard) Let $g: \R \to (0,+\infty)$ be a 1-periodic function, and assume that $g \in L_1(0,1)$. Prove that if $f_n \to 0$ a.e. on $(0,1)$, and
    \[
     |f_n(x)| \le g(nx) \quad \text{for all } x \in (0,1),
    \]
    then $\int_0^1f_n (x) \, dx \to 0$.
\end{problem}

\end{document}
% Local Variables:
% eval: (LaTeX-add-environments '("problem" "Date" "Tags"))
% End:
