\documentclass[11pt, notitlepage]{article}
\usepackage[utf8]{inputenc}

% \usepackage[nottoc]{tocbibind}
\usepackage{datetime}
\usepackage{graphicx}
\usepackage[nointegrals]{wasysym}
\usepackage{mathtools}
\usepackage{tikz}
\usepackage{tikz-cd}
\usepackage{marginnote}
\usepackage{microtype} % Slightly tweak font spacing for aesthetics
\usepackage{amsmath, amsfonts, amssymb, amsthm}
\usepackage{bbm}
% \usepackage{thm-autoref}
\usepackage{hyperref}
\hypersetup{
  colorlinks   = true, %Colours links instead of ugly boxes
  urlcolor     = blue, %Colour for external hyperlinks
  linkcolor    = blue, %Colour of internal links
  citecolor   = red %Colour of citations
}

\newtheorem{thm}{Theorem}[section]

\newtheorem{lem}[thm]{Lemma}
% \newcommand{\lemautorefname}{Lemma}

\newtheorem{claim}[thm]{Claim}
\newtheorem{prop}[thm]{Proposition}
\newtheorem{cor}[thm]{Corollary}
\newtheorem{conj}[thm]{Conjecture}

\theoremstyle{definition}
\newtheorem{defn}[thm]{Definition}

\theoremstyle{definition}
\newtheorem{example}[thm]{Example}

\theoremstyle{definition}
\newtheorem*{question}{Question}
\newtheorem*{exercise}{Exercise}
\newtheorem{probstate}{Problem}

\theoremstyle{remark}
\newtheorem*{rem}{Remark}

\usepackage[top=2cm, bottom=2cm, inner=2cm, outer=4cm, marginparwidth=3.4cm]{geometry}

\usepackage{gentium}
%\usepackage[euler-digits,small]{eulervm}

\usepackage{enumerate}
\usepackage{xcolor}

%-------------------------------
%       Custom environments and counters
%-------------------------------

%\newcounter{problemnum}
%\setcounter{problemnum}{1}

\newenvironment{problem}[2]{
    \begin{probstate}[Date: #1, tags: \texttt{#2}]
}
{
  \end{probstate}
}

\newenvironment{solution}{
    \begin{proof}[Solution]
}
{
    \end{proof}
}

\title{Real Analysis qualifying review big list}
\author{Students at the University of Michigan}
% \date{\formatdate{3}{10}{2019}}

\makeatletter
\makeatother

\newcommand{\R}{\mathbb{R}}

\begin{document}
\maketitle

%\tableofcontents

\section{Categorized (without solutions)}

\begin{problem}{September 2011}{integral convergence}
Let $f \in L_1([0,1], dx)$. Find
 \[
   \lim_{n \to \infty} \frac{1}{n}  \int_0^1 \log \left( 1+ e^{nf(x)} \right) \, dx.
 \]
\end{problem}

\begin{problem}{September 2011}{bounded variation}
  Suppose $f:[0,1] \to \R$ satisfies $f(x)-f(y)<x-y$ for all $x,y\in [0,1], x>y$.
  Show that $f'$ exists almost everywhere on $[0,1]$ or give a counterexample.
\end{problem}

\begin{problem}{September 2011}{hölder's inequality}
  Let $(X,\Omega, \mu)$ be a finite measure space.
 \begin{enumerate}[(i)]
   \item Prove that for any $p<q, \ L_q(\mu) \subset L_p(\mu)$.
   \item Assume that for any
 $t>0$ there exists $E \in \Omega$ satisfying
 \[
  0< \mu(E) <t.
 \]
 Prove that for any $1<p< \infty$ there exists a function $f \in
 L_p(\mu)$ such that $f \notin L_q(\mu)$ for any $q>p$.
\end{enumerate}
\end{problem}

\begin{problem}{September 2011}{$L^p$ spaces}
  For a real valued function $f(x,y)$ on $\mathbb R^2$ which is in $L^2$, show that $f(x+\epsilon,y+\epsilon) \rightarrow f(x,y)$ in $L^2$ when $\epsilon \rightarrow 0$.
\end{problem}


\begin{problem}{September 2011}{measurable sets?}
  Let $f \in L_1([0,1])$ be a function such that $\int_E f (x) \, dx=0$ for any measurable set $E \subset [0,1]$ of Lebesgue measure $1/2$.
  Prove that $f=0$ a.e.
\end{problem}

\begin{problem}{January 2011}{measurable functions}
  \item Let A be a sequence of measurable subsets of $[0,1]$ such that $\inf m(A_n)>0$, where $m$ stands for the Lebesgue measure.
   \begin{enumerate}[(i)]
    \item Prove that there exists $x \in [0,1]$ which belongs to infinitely many of the sets $A_n$.
    \item Does there necessarily exist a point which {\color{red} (does not?)} belong to any of the sets $A_n$, except finitely many?
   \end{enumerate}
\end{problem}

\begin{problem}{January 2011}{integral convergence}
  Let $\{f_n\} \subset L_1(\mu)$ be a decreasing sequence of functions such that $f_n \to f$ a.e.
  Prove that
    \[
      \lim_{n \to \infty} \int f_n \, d \mu = \int f \, d \mu.
    \]
\end{problem}

\begin{problem}{January 2011}{simple function approximation}
  Let $f:X \to [0, +\infty)$ be an integrable function on
    a measure space $(X, \mathcal{A},\mu)$.
    Define the measure $\nu$ by $\nu(A)=\int_A f \, d \mu$.
    \begin{enumerate}[(i)]
        \item Prove that the measure $\nu$ is $\sigma$-additive.

        \item Prove that if
        $g \in L_1(\nu)$, then
        $\int_X g \, d \nu =\int_X fg \, d \mu$.

        (Hint: first, assume that $g$ is a simple positive
        function. Then extend the result to non-negative
        integrable functions using limit theorems).
    \end{enumerate}
\end{problem}

\begin{problem}{January 2011}{integral inequalities}
  Let $f_n: \mathbb{R} \to [0, 1]$ be functions such that
 $\sup_{x \in \mathbb{R}} f_n(x)=1/n$ and $\int_{\mathbb{R}} f(x) \, dx=1$. Set
 \[
  F(x)= \sup_{n \in \mathbb{N}} f_n(x).
 \]
 Find all possible values of $\int_{\mathbb{R}} F(x) \, dx$.

\end{problem}

\begin{problem}{January 2011}{integral convergence?}
  Let $f \in L_{\infty}([0,1])$. Prove that
 \[
   \lim_{n \to \infty} \frac{\int_{[0,1]} |f(x)|^{n+1} \, dx}{\int_{[0,1]} |f(x)|^{n} \, dx} = \|f\|_{\infty}.
 \]

\end{problem}

\begin{problem}{January 2011}{egoroff's theorem}
  Let $E$ be the exceptional set in Egoroff's theorem.
    Is it possible to prove Egoroff's theorem with
    $\l(E)=0$ instead of $\l(E) < e$?
\end{problem}

\begin{problem}{January 2011}{simple functions}
  Let $f:X \to [0, \infty]$ be a measurable
 function. Assume that $\mu (X) < \infty$.
 Prove that $\int f \, d \mu< +\infty$ if and only if
 \[
 \sum_{n=1}^{\infty} 2^n \mu (x \in X \mid f(x) \ge 2^n) < +\infty.
 \]
\end{problem}

\begin{problem}{January 2011}{dominated convergence}
  Let $f_n,g_n,f,g \in L_1(\mu)$ be functions such that
    $f_n \to f$ a.e., $g_n \to g$ a.e. and $|f_n| \le g_n$. Prove
    that if $\int g_n \, d \mu \to \int g \, d \mu$, then
    $\int f_n \, d \mu \to \int f \, d \mu$.

    (Hint: follow the proof of Lebesgue dominated convergence
    theorem.)
\end{problem}

\begin{problem}{January 2011}{}
  Let $f_n,f \in L_1(\mu)$ be such that $f_n \to f$ a.e.
    Prove that if \newline
    $\|f_n\|_1 \to \|f\|_1$,
    then $f_n \to f$ in $L_1(\mu)$.

    (Hint: use the previous problem)

\end{problem}


\begin{problem}{January 2011}{change of variables}
  Let $f,g: \mathbb R \rightarrow \mathbb R$ be $L_1$-functions.
  \begin{enumerate}[(a)]
\item Prove that
\[
\int_{\mathbb R} |f(x-y)g(y)| dm(y) < +\infty.
\]
  \item Let
\[
h(x)= \int_{\mathbb R} f(x-y)g(y) dm(y).
\]
Prove that $h \in L_1(\mathbb R)$ and $\|h\|_1 \le \|f\|_1 \cdot \|g\|_1$.
\end{enumerate}
\end{problem}


\section{Categorized (with solutions)}

\section{Uncategorized}

\begin{problem}{January 2011}{}
  Let $f \in L_1([0,1])$ be a function such that $f(x)>0$ a.e.
 \begin{enumerate}[(i)]
 \item Prove that for any $0<a<1$
 \[
   \inf_{m(A)=a} \int_A f \, dm >0.
 \]
 \item Does the previous statement hold for a function $f \in L_1(\mathbb R)$ such that $f(x)>0$ a.e.
 \end{enumerate}
\end{problem}

\begin{problem}{January 2011}{}
 Let $(X,\Omega, \mu)$ be a finite measure space.
 \begin{enumerate}[(i)]
   \item Prove that for any $p<q, \ L_q(\mu) \subset L_p(\mu)$.
   \item Assume that for any
 $t>0$ there exists $E \in \Omega$ satisfying
 \[
  0< \mu(E) <t.
 \]
 Prove that for any $1<p< \infty$ there exists a function $f \in
 L_p(\mu)$ such that $f \notin L_q(\mu)$ for any $q>p$.

 \end{enumerate}
\end{problem}

\end{document}
% Local Variables:
% eval: (LaTeX-add-environments '("problem" "Date" "Tags"))
% End:
